\documentclass[12pt, letterpaper]{article}
%\documentclass[12pt, letterpaper, titlepage]{article}

\usepackage{amsmath}
\usepackage{booktabs}
\usepackage{amsthm}
\usepackage{graphicx}
\usepackage[margin=1in]{geometry}
\usepackage{hyperref}
\hypersetup{colorlinks = true, linkcolor = blue, citecolor=blue, urlcolor = blue}
\usepackage{natbib}
\usepackage{enumitem}
\usepackage{setspace}

\usepackage[]{lineno}
\linenumbers*[1]
% %% patches to make lineno work better with amsmath
\newcommand*\patchAmsMathEnvironmentForLineno[1]{%
 \expandafter\let\csname old#1\expandafter\endcsname\csname #1\endcsname
 \expandafter\let\csname oldend#1\expandafter\endcsname\csname end#1\endcsname
 \renewenvironment{#1}%
 {\linenomath\csname old#1\endcsname}%
 {\csname oldend#1\endcsname\endlinenomath}}%
\newcommand*\patchBothAmsMathEnvironmentsForLineno[1]{%
 \patchAmsMathEnvironmentForLineno{#1}%
 \patchAmsMathEnvironmentForLineno{#1*}}%

\AtBeginDocument{%
 \patchBothAmsMathEnvironmentsForLineno{equation}%
 \patchBothAmsMathEnvironmentsForLineno{align}%
 \patchBothAmsMathEnvironmentsForLineno{flalign}%
 \patchBothAmsMathEnvironmentsForLineno{alignat}%
 \patchBothAmsMathEnvironmentsForLineno{gather}%
 \patchBothAmsMathEnvironmentsForLineno{multline}%
}

% control floats
\renewcommand\floatpagefraction{.9}
\renewcommand\topfraction{.9}
\renewcommand\bottomfraction{.9}
\renewcommand\textfraction{.1}
\setcounter{totalnumber}{50}
\setcounter{topnumber}{50}
\setcounter{bottomnumber}{50}

% Definition of author-specific comments
\usepackage{xcolor}
\newcommand{\jy}[1]{\textcolor{orange}{JY: #1}}

% NOTE: To produce blinded version, replace "0" with "1" below.
\newcommand{\blind}{0}


%\title{On Misuses of the Kolmogorov--Smirnov Test for One-Sample Goodness-of-Fit}
%
%\author{Anthony Zeimbekakis\\
%%   \href{mailto:anthony.zeimbekakis@uconn.edu}
%% {\nolinkurl{anthony.zeimbekakis@uconn.edu}}\\
  %Elizabeth D.  Schifano\\
  %Jun Yan\\[1ex]
  %Department of Statistics, University of Connecticut\\
%}
%\date{}

\begin{document}
%\maketitle

\title{\bf Data Jamboree: A Party of Different Computing Tools Solving the
  Same Data Science Problems}
\if0\blind
{
  \author{Lucy D'Agostino McGowan,
    Shannon Tass,
    Sam Tyner,\\
    HaiYing Wang,
    Jun Yan
  }
} \fi

\maketitle

\doublespace

\begin{abstract}

  \jy{The title is tentative. Needs to reflect 1) three languages; 2) case study.}


\bigskip
\noindent{\sc Keywords}:
data science;
education;
portability;
statistical computing
\end{abstract}

%\doublespace

\section{Introduction}
\label{sec:intro}


\jy{These paragraphs are place holders. Please add points that should appear in
  the introduction.}

\jy{Please discuss what we should include in the paper. We shall refer to code
  in GitHub repos. What're the things that we'd like to highlight in the paper
  for each language? Should we create a table to compare the pros and cons?}

\jy{My thoughts are to present the advantages and limitations of each language
  in the setting of this case study under each section. At the end, we can
  discuss at a higher level about the languages, the jamboree, and open data in
  statistics/data science education.}


% Importance of Computing in Statistics and Data Science Education
The evolving landscape of statistics and data science education increasingly
emphasizes the critical role of computing. This shift, as detailed by
\citet{hardin2021computing} and \citet{nolan2010computing}, reflects a broader
trend towards integrating computational methods with traditional statistical
techniques. These skills are indispensable in modern data analysis, enabling
students to effectively handle and interpret complex
datasets. \citet{hicks2018guide} further advocate for a comprehensive teaching
approach in data science, one that seamlessly blends computational tools with
statistical concepts. The inclusion of programming languages like R, Python, and
Julia in curricula is not just about tool proficiency; it is about fostering a
deeper understanding of data analysis and statistical inference in a digitally
driven world.


% Various Forms of Data Competitions and Data Fests
Data competitions and fests, such as hackathons \citep{lara2016hackathons} and
DataFests \citep{noll2023insights}, represent vital practical components in the
education of future data scientists and statisticians. While hackathons
typically focus on creating a working prototype or solution in a short time,
often with a strong emphasis on coding and immediate problem-solving, DataFests
usually involve more in-depth analysis of large datasets, focusing on
statistical insights and data visualization. These events challenge
participants to apply their skills in real-world scenarios, encouraging not just
the application of statistical and computational knowledge but also the
development of collaborative and innovative problem-solving
abilities. Hackathons and DataFests serve as platforms for experiential
learning, where the theoretical knowledge gained in the classroom is tested and
expanded upon in a dynamic and often competitive environment. These experiences
are invaluable in preparing students for the realities of professional roles in
data science, where teamwork, creativity, and ethical considerations in data
usage are just as important as technical prowess.


% Defining the Data Jamboree
The Data Jamboree is an innovative event that brings together diverse computing
tools to collaboratively tackle data science problems using a shared dataset. As
exemplified by the recent jamboree, which utilized the NYC Open Data of 311
Service Requests, it provides a unique platform where participants engage in
real-world data challenges within a specified timeframe. A Data Jamboree
emphasizes the use of various computing tools to solve the same data science
problems, fostering an environment where the comparative strengths of different
programming languages and approaches (like R, Python, and Julia) can be explored
and evaluated. This approach highlights the strengths and particularities of each
language in solving real-world challenges. The event's format,
focusing on data cleaning, manipulation, and analysis, is meticulously designed
to mirror the complexities and nuances of real-world data science tasks, as
understood in contemporary data science education
\citep{nolan2010computing}. This approach not only enhances practical skills in
data science but also fosters a collaborative and competitive environment,
encouraging innovation, critical thinking, and teamwork among participants, as
noted by \citet{dalzell2023increasing}.



% Advantages of Using Open Data
The utilization of open data in the Data Jamboree, especially in the context of
different programming languages, presents a significant educational and
practical advantage. Open data initiatives, as detailed by
\citet{beheshti2019datasynapse} and \citet{janssen2012benefits}, promote
transparency and public engagement, crucial in data science. The use of open
data allows participants to apply R, Python, and Julia in real-world scenarios,
demonstrating how different languages can be used to manipulate and analyze the
same datasets. This approach aligns with \citet{borgman2012conundrum} and
\citet{cantor2018facets} on open data's role in fostering diverse data-driven
solutions and contributes to the discourse on using data for societal benefits,
as discussed in \citet{ridgway2023data}.


\section{The Arena}
\label{sec:arena}

The NYC 311 Service Requests dataset represents a comprehensive accumulation of
non-emergency requests made by New York City residents. The dataset includes a
wide array of urban issues, ranging from noise complaints and pothole reporting
to graffiti removal and street light issues. Since its inception in 2010, the
dataset has served as a rich resource for understanding urban living dynamics
and for public service improvement. It reflects the day-to-day concerns of New
Yorkers, providing insight into the spatial and temporal patterns of urban
issues. This open dataset's accessibility makes it an excellent resource for
data science projects and competitions, offering real-world relevance and a wide
scope for analysis.


For the jamboree, a subset of requests was considered, those created between
January 15 and 21, 2023, showcasing real-time urban issues ranging from
infrastructural complaints to community concerns.


The scientific exercises of the jamboree were divided into three main stages:

\emph{Data Cleaning:} This stage required participants to standardize column
names across different programming languages for ease of comparison, identify
and correct errors or inefficiencies (e.g., closed dates earlier than created
dates, invalid values), and handle missing values (e.g., using geocoding to fill
missing zip codes). Participants were also asked to summarize their suggestions
for the data curator.

\emph{Data Manipulation:} Focusing on NYPD requests, this stage involved
creating a new variable, 'duration', to represent the time period from the
Created Date to the Closed Date. Participants visualized the distribution of
uncensored duration by weekdays/weekends and boroughs, tested distribution
similarities, and merged zipcode-level information such as population density,
home values, and household income from the US Census with the NYPD requests
data.

\emph{Data Analysis} The final stage involved defining a binary variable
'over3h' for requests with durations over three hours, building a logistic model
to predict this variable using 311 request data and zipcode-level covariates,
and repeating the analysis with different models like random forests or neural
networks.

See \url{https://asa-ssc.github.io/minisymp2023/jamboree/} for more specifics
about the requirements.



\section{The Performances}
\label{sec:perf}

\jy{Use this section to compare the three languages in the jamboree setting;
  highlight language-specific goodies; link to comparisons on the languages in
  the literature.}


\section{Design Your Own Jamboree}
\label{sec:design}

\jy{Use this section to help data science event organizers design a Jamboree;
  would SAS be a possibility? I'd not endorse any commercial software, but am
  open to others' opinions.}

\jy{If we get interesting results from the post-event survey, we can also
  present them here.}

\section{Discussion}
\label{sec:disc}

\jy{Things to discuss:
  Comparison of three languages;
  Generalizability of data jamboree in datafests or other educational events;
  Importance of working on real-world problems made possible by open data.
  etc.}



\bibliographystyle{asa}
\bibliography{citations}


\end{document}

%%% LocalWords: nonparametric semiparametric autocorrelation ARMA
%%% Local Variables:
%%% mode: latex
%%% TeX-master: t
%%% ispell-personal-dictionary: ".aspell.en.pws"
%%% fill-column: 80
%%% eval: (auto-fill-mode 1)
%%% End:
